\nonstopmode{}
\documentclass[a4paper]{book}
\usepackage[times,inconsolata,hyper]{Rd}
\usepackage{makeidx}
\usepackage[utf8,latin1]{inputenc}
% \usepackage{graphicx} % @USE GRAPHICX@
\makeindex{}
\begin{document}
\chapter*{}
\begin{center}
{\textbf{\huge Package `swiTheme'}}
\par\bigskip{\large \today}
\end{center}
\begin{description}
\raggedright{}
\item[Type]\AsIs{Package}
\item[Title]\AsIs{ggplot2 swissinfo.ch theme}
\item[Version]\AsIs{0.5}
\item[Date]\AsIs{2014-11-09}
\item[Author]\AsIs{Duc-Quang Nguyen}
\item[Maintainer]\AsIs{Duc-Quang Nguyen }\email{ducquang.nguyen@swssinfo.ch}\AsIs{}
\item[Imports]\AsIs{ggplot2 (>= 2.1.0), scales, extrafont, grid, gridExtra}
\item[Description]\AsIs{More about what it does (maybe more than one line)}
\item[License]\AsIs{GNU}
\item[RoxygenNote]\AsIs{5.0.1}
\item[NeedsCompilation]\AsIs{no}
\end{description}
\Rdcontents{\R{} topics documented:}
\inputencoding{utf8}
\HeaderA{swiTheme-package}{What the package does (short line) \textasciitilde{}\textasciitilde{} package title \textasciitilde{}\textasciitilde{}}{swiTheme.Rdash.package}
\aliasA{swiTheme}{swiTheme-package}{swiTheme}
\keyword{ggplot2, pdf, grid, scales}{swiTheme-package}
%
\begin{Description}\relax
More about what it does (maybe more than one line)
\textasciitilde{}\textasciitilde{} A concise (1-5 lines) description of the package \textasciitilde{}\textasciitilde{}
\end{Description}
%
\begin{Details}\relax

\Tabular{ll}{
Package: & swiTheme\\{}
Type: & Package\\{}
Version: & 1.0\\{}
Date: & 2014-11-09\\{}
License: & What license is it under?\\{}
}
\textasciitilde{}\textasciitilde{} An overview of how to use the package, including the most important functions \textasciitilde{}\textasciitilde{}
\end{Details}
%
\begin{Author}\relax
Who wrote it

Maintainer: Who to complain to <yourfault@somewhere.net>
\textasciitilde{}\textasciitilde{} The author and/or maintainer of the package \textasciitilde{}\textasciitilde{}
\end{Author}
%
\begin{References}\relax
\textasciitilde{}\textasciitilde{} Literature or other references for background information \textasciitilde{}\textasciitilde{}
\end{References}
%
\begin{SeeAlso}\relax
\textasciitilde{}\textasciitilde{} Optional links to other man pages, e.g. \textasciitilde{}\textasciitilde{}
\textasciitilde{}\textasciitilde{} \code{\LinkA{<pkg>}{<pkg>}} \textasciitilde{}\textasciitilde{}
\end{SeeAlso}
%
\begin{Examples}
\begin{ExampleCode}
~~ simple examples of the most important functions ~~
\end{ExampleCode}
\end{Examples}
\inputencoding{utf8}
\HeaderA{multiplot}{Multiple plot function}{multiplot}
%
\begin{Description}\relax
To arrange multiple ggplot chart on a grid. Copied from: http://www.cookbook-r.com/Graphs/Multiple\_graphs\_on\_one\_page\_(ggplot2)/
\end{Description}
%
\begin{Usage}
\begin{verbatim}
multiplot(plotlist = NULL, cols = 1, layout = NULL)
\end{verbatim}
\end{Usage}
%
\begin{Arguments}
\begin{ldescription}
\item[\code{cols}] Number of columns in layout

\item[\code{layout}] A matrix specifying the layout. If present, 'cols' is ignored.

\item[\code{plots}] a list of ggplot objects
\end{ldescription}
\end{Arguments}
%
\begin{Details}\relax
If the layout is something like matrix(c(1,2,3,3), nrow=2, byrow=TRUE), then plot 1 will go in the upper left, 2 will go in the upper right, and ll go all the way across the bottom.
\end{Details}
%
\begin{Examples}
\begin{ExampleCode}
q1 <- qplot(1:10, 1:10, size = 10:1) + xlab("axis x label") + ylab ("y axis label") + theme_swi2()
q2 <- qplot(mpg, data = mtcars, geom = "dotplot") + theme_swi()
multiplot(list(q1, q2))
\end{ExampleCode}
\end{Examples}
\inputencoding{utf8}
\HeaderA{pdfswi\_sq}{swissinfo standard chart size pdf export}{pdfswi.Rul.sq}
\aliasA{pdfswi\_long}{pdfswi\_sq}{pdfswi.Rul.long}
%
\begin{Description}\relax
Simple wrapper for R's graphics device driver (i.e. pdf) to save charts in pre-defined sizes
\end{Description}
%
\begin{Usage}
\begin{verbatim}
pdfswi_sq(file = "", width = 10, height = 10, useDingbats = FALSE, ...)

pdfswi_long(file = "", width = 6, height = width * 1.25,
  useDingbats = FALSE, ...)
\end{verbatim}
\end{Usage}
\inputencoding{utf8}
\HeaderA{swi\_pal}{swissinfo.ch standard color palette}{swi.Rul.pal}
\aliasA{swi\_dpal}{swi\_pal}{swi.Rul.dpal}
\aliasA{swi\_dpal2}{swi\_pal}{swi.Rul.dpal2}
\aliasA{swi\_rpal}{swi\_pal}{swi.Rul.rpal}
\aliasA{swi\_spal}{swi\_pal}{swi.Rul.spal}
\keyword{datasets}{swi\_pal}
%
\begin{Description}\relax
swissinfo.ch standard color palette

swissinfo random color palette
\end{Description}
%
\begin{Usage}
\begin{verbatim}
swi_pal

swi_rpal

swi_dpal

swi_dpal2

swi_spal
\end{verbatim}
\end{Usage}
%
\begin{Format}
An object of class \code{character} of length 22.
\end{Format}
%
\begin{Examples}
\begin{ExampleCode}
pie(rep(1,length(swi_pal)), col=swi_pal)
pie(rep(1,length(swi_rpal)), col=swi_rpal)
pie(rep(1,length(swi_dpal)), col=swi_dpal)
pie(rep(1,length(swi_dpal)), col=swi_dpal2)
pie(rep(1,length(swi_spal)), col=swi_spal)
\end{ExampleCode}
\end{Examples}
\inputencoding{utf8}
\HeaderA{theme\_swi}{swissinfo.ch's chart theme}{theme.Rul.swi}
\aliasA{swi\_theme}{theme\_swi}{swi.Rul.theme}
\aliasA{theme\_swi2}{theme\_swi}{theme.Rul.swi2}
\aliasA{theme\_swiMin}{theme\_swi}{theme.Rul.swiMin}
\aliasA{theme\_swiYLines}{theme\_swi}{theme.Rul.swiYLines}
%
\begin{Description}\relax
swi\_theme(): ggplot copied/inspired from https://gist.github.com/hrbrmstr/283850725519e502e70c

swissinfo minimal font and color ggplot2 theme

theme\_swiYLines: a ggplot2 theme with horizontal grid lines

theme\_swi2, with grid lines ggplot2 minimal theme

theme\_swiMin, a minimal theme with no axis, no ticks, no axis text, no grid lines
\end{Description}
%
\begin{Usage}
\begin{verbatim}
swi_theme(base_size = 12, base_family = "OpenSans-CondensedLight",
  title_family = "OpenSans-Light",
  subtitle = "OpenSans-CondensedLightItalic", y_gridlines = TRUE,
  base_color = "#2b2b2b")

theme_swi(ticks = TRUE, base_family = "Open Sans", base_size = 11)

theme_swiYLines(yaxis = FALSE, base_family = "Open Sans",
  base_family2 = "Open Sans Semibold", base_size = 11,
  axisColor = "#7E8279")

theme_swi2(base_family = "Open Sans", base_family2 = "Open Sans Semibold",
  base_size = 11, axisColor = "#7E8279")

theme_swiMin(base_family = "Open Sans", base_size = 11)
\end{verbatim}
\end{Usage}
%
\begin{Arguments}
\begin{ldescription}
\item[\code{base\_size}] Base font size

\item[\code{base\_family}] Base font family

\item[\code{y\_gridlines}] logical, display y gridlines

\item[\code{ticks}] \code{logical} Show axis ticks?

\item[\code{base\_family2}] secondary font family

\item[\code{axisColor}] the color for the axis and their ticks and labels

\item[\code{base\_family}] Base font family

\item[\code{axisColor}] the color for the axis and their ticks and labels

\item[\code{base\_family2}] secondary font family
\end{ldescription}
\end{Arguments}
%
\begin{Examples}
\begin{ExampleCode}
# swi_theme() with annotations
gg <- qplot(1:10, 1:10) + swi_theme()
# add y-axis label 
gg <- gg + geom_label(aes(x = 0, y = 10, label="y label text"),
  family = "OpenSans-CondensedLight", size=3.5, 
  hjust=0, label.size=0, color="#2b2b2b")
# basic axis scale and breaks
gg <- gg + scale_x_continuous(expand=c(0,0), breaks = seq(0, 10, by=2), 
  labels = seq(0, 10, by=2), limits=c(0, 10.2), name = NULL)
gg <- gg + scale_y_continuous(expand=c(0,0.3), breaks=seq(1, 10, by=1), limits=c(0, 10.2))
# footnote / caption / subtitle
caption <- "Note: Vacancies are counted as the number of days between a justice's death, retirement or resignation and the successor justice's swearing in (or commissioning in the case of a recess appointment) as a member of the court.Sources: U.S. Senate, 'Supreme Court Nominations, present-1789'; Supreme Court, 'Members of the Supreme Court of the United States'; Pew Research Center calculations"
caption  <- paste0(strwrap(caption, 160), sep="", collapse="\n")
subtitle <- "Here is a subtitle text. It describes what is displayed in the chart. It can mention many important details to under the graphic."
subtitle  <- paste0(strwrap(subtitle, 160), sep="", collapse="\n") 
gg <- gg + labs(x=NULL, y=NULL, title="This is a title, choose it wisely",
subtitle=subtitle, caption=caption)
#annotations TODO !!!!

qplot(1:10, 1:10, size = 10:1) + xlab("axis x label") + ylab ("y axis label") + theme_swi()
qplot(mtcars$mpg) + theme_swi()


qplot(1:10, 1:10, size = 10:1) + xlab("axis x label") + ylab ("y axis label") + theme_swiYLines()
qplot(1:10, 1:10, size = 10:1) + xlab("axis x label") + ylab ("y axis label") + theme_swi2()
 qplot(1:10, 1:10, size = 10:1) + theme_swiMin() + ggtitle("Minimal theme: no axis & no ticks")
\end{ExampleCode}
\end{Examples}
\printindex{}
\end{document}
